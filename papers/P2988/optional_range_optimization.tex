\documentclass[a4paper,10pt,oneside,openany,final,article]{memoir}
\input{common}
\settocdepth{chapter}
\usepackage{minted}
\usepackage{fontspec}
\setromanfont{Source Serif Pro}
\setsansfont{Source Sans Pro}
% \setmonofont{Source Code Pro}

\begin{document}
\title{Optimize for std::optional in range adaptors}
\author{
  Steve Downey \small<\href{mailto:sdowney@gmail.com}{sdowney@gmail.com}> \\
  Tomasz Kamiński \small<\href{mailto:tomaszkam@gmail.com}{tomaszkam@gmail.com}> \\
}
\date{} %unused. Type date explicitly below.
\maketitle

\begin{flushright}
  \begin{tabular}{ll}
    Document \#: & D3913R0 \\
    Date: & \today \\
    Project: & Programming Language C++ \\
    Audience: & LEWG,LWG
  \end{tabular}
\end{flushright}

\begin{abstract}
From PL-011 22.5 [optional] Optimize for std::optional in range adaptors

The range support was added to the optional, making it usable with range adaptors defined in std::views, however, we have not updated the views specification to handle it optimally when possible. This leads to unnecessary template instantiations.

\end{abstract}

\tableofcontents*

\chapter{Motivation}
The range support was added to the optional, making it usable with range adaptors defined in std::views, however, we have not updated the views specification to handle it optimally when possible. This leads to unnecessary template instantiations.

Proposed change:

Add a special case to recognize optional for adaptors:

views::as_const: should return optional or optional<const U\&> (if T is U\&)
views::take(opt, n): empty optional if n is equal to zero, opt otherwise
views::drop(opt, n): empty optional if n greater than zero, opt otherwise
views::reverse: input unchanged


\chapter{Design}


\section{views::as_const}

return optional or optional<const U\&> (if T is U\&)

\section{views::take(opt, n)}

empty optional if n is equal to zero, opt otherwise

\section{views::drop(opt, n)}

empty optional if n greater than zero, opt otherwise

\section{views::reverse}

input unchanged

\chapter{Wording}
The proposed changes are relative to the current working draft \cite{N5014}.


\begin{wording}
\rSec1[ranges.general]{General}

\rSec2[range.take]{Take view}

\rSec3[range.take.overview]{Overview}

\pnum
\tcode{take_view} produces a view of the first $N$ elements
from another view, or all the elements if the adapted
view contains fewer than $N$.

\pnum
\indexlibrarymember{take}{views}%
The name \tcode{views::take} denotes a
range adaptor object\iref{range.adaptor.object}.
Let \tcode{E} and \tcode{F} be expressions,
let \tcode{T} be \tcode{remove_cvref_t<decltype((E))>}, and
let \tcode{D} be \tcode{range_difference_t<decltype((E))>}.
If \tcode{decltype((F))} does not model
\tcode{\libconcept{convertible_to}<D>},
\tcode{views::take(E, F)} is ill-formed.
Otherwise, the expression \tcode{views::take(E, F)}
is expression-equivalent to:

\begin{itemize}
\item
If \tcode{T} is a specialization
of \tcode{empty_view}\iref{range.empty.view},
then \tcode{((void)F, \placeholdernc{decay-copy}(E))},
except that the evaluations of \tcode{E} and \tcode{F}
are indeterminately sequenced.
\begin{addedblock}
\item
Otherwise, If \tcode{T} is specialization of \tcode{optional}, then \tcode{(F == D(0) ?  ((void)E, T()) : \placeholdernc{decay-copy}(E))}
\end{addedblock}
\item
Otherwise, if \tcode{T} models
\libconcept{random_access_range} and \libconcept{sized_range}
and is a specialization of
\tcode{span}\iref{views.span},
\tcode{basic_string_view}\iref{string.view}, or
\tcode{subrange}\iref{range.subrange},
then
\tcode{U(ranges::begin(E),
ranges::be\-gin(E) + std::min<D>(ranges::distance(E), F))},
except that \tcode{E} is evaluated only once,
where \tcode{U} is a type determined as follows:

\begin{itemize}
\item if \tcode{T} is a specialization of \tcode{span},
then \tcode{U} is \tcode{span<typename T::element_type>};
\item otherwise, if \tcode{T} is a specialization of \tcode{basic_string_view},
then \tcode{U} is \tcode{T};
\item otherwise, \tcode{T} is a specialization of \tcode{subrange}, and
\tcode{U} is \tcode{subrange<iterator_t<T>>};
\end{itemize}

\item
otherwise, if \tcode{T} is
a specialization of \tcode{iota_view}\iref{range.iota.view}
that models \libconcept{random_access_range} and \libconcept{sized_range},
then
\tcode{iota_view(*ranges::begin(E),
*(ranges::begin(E) + std::\linebreak{}min<D>(ranges::distance(E), F)))},
except that \tcode{E} is evaluated only once.

\item
Otherwise, if \tcode{T} is
a specialization of \tcode{repeat_view}\iref{range.repeat.view}:
\begin{itemize}
\item
if \tcode{T} models \libconcept{sized_range},
then
\begin{codeblock}
views::repeat(*E.@\exposid{value_}@, std::min<D>(ranges::distance(E), F))
\end{codeblock}
except that \tcode{E} is evaluated only once;
\item
otherwise, \tcode{views::repeat(*E.\exposid{value_}, static_cast<D>(F))}.
\end{itemize}

\item
Otherwise, \tcode{take_view(E, F)}.
\end{itemize}

\rSec2[range.drop]{Drop view}

\rSec3[range.drop.overview]{Overview}

\pnum
\tcode{drop_view} produces a view
excluding the first $N$ elements from another view, or
an empty range if the adapted view contains fewer than $N$ elements.

\pnum
\indexlibrarymember{drop}{views}%
The name \tcode{views::drop} denotes
a range adaptor object\iref{range.adaptor.object}.
Let \tcode{E} and \tcode{F} be expressions,
let \tcode{T} be \tcode{remove_cvref_t<decltype((E))>}, and
let \tcode{D} be \tcode{range_difference_t<decltype((E))>}.
If \tcode{decltype((F))} does not model
\tcode{\libconcept{convertible_to}<D>},
\tcode{views::drop(E, F)} is ill-formed.
Otherwise, the expression \tcode{views::drop(E, F)}
is expression-equivalent to:

\begin{itemize}
\item
If \tcode{T} is a specialization of
\tcode{empty_view}\iref{range.empty.view},
then \tcode{((void)F, \placeholdernc{decay-copy}(E))},
except that the evaluations of \tcode{E} and \tcode{F}
are indeterminately sequenced.
\begin{addedblock}
\item
Otherwise, If \tcode{T} is specialization of \tcode{optional}, then \tcode{(F == D(0) ? \placeholdernc{decay-copy}(E) : ((void)E, T()))}
\end{addedblock}
\item
Otherwise, if \tcode{T} models
\libconcept{random_access_range} and \libconcept{sized_range}
and is
\begin{itemize}
\item a specialization of \tcode{span}\iref{views.span},
\item a specialization of \tcode{basic_string_view}\iref{string.view},
\item a specialization of \tcode{iota_view}\iref{range.iota.view}, or
\item a specialization of \tcode{subrange}\iref{range.subrange}
where \tcode{T::\exposid{StoreSize}} is \tcode{false},
\end{itemize}
then \tcode{U(ranges::begin(E) + std::min<D>(ranges::distance(E), F), ranges::end(E))},
except that \tcode{E} is evaluated only once,
where \tcode{U} is \tcode{span<typename T::element_type>}
if \tcode{T} is a specialization of \tcode{span} and \tcode{T} otherwise.

\item
Otherwise,
if \tcode{T} is
a specialization of \tcode{subrange}
that models \libconcept{random_access_range} and \libconcept{sized_range},
then
\tcode{T(ranges::begin(E) + std::min<D>(ranges::distance(E), F), ranges::\linebreak{}end(E),
\exposid{to-unsigned-like}(ranges::distance(E) -
std::min<D>(ranges::distance(E), F)))},
except that \tcode{E} and \tcode{F} are each evaluated only once.

\item
Otherwise, if \tcode{T} is
a specialization of \tcode{repeat_view}\iref{range.repeat.view}:
\begin{itemize}
\item
if \tcode{T} models \libconcept{sized_range},
then
\begin{codeblock}
views::repeat(*E.@\exposid{value_}@, ranges::distance(E) - std::min<D>(ranges::distance(E), F))
\end{codeblock}
except that \tcode{E} is evaluated only once;
\item
otherwise, \tcode{((void)F, \placeholdernc{decay-copy}(E))},
except that the evaluations of \tcode{E} and \tcode{F} are indeterminately sequenced.
\end{itemize}

\item
Otherwise, \tcode{drop_view(E, F)}.
\end{itemize}

\rSec2[range.as.const]{As const view}

\rSec3[range.as.const.overview]{Overview}

\pnum
\tcode{as_const_view} presents a view of an underlying sequence as constant.
That is, the elements of an \tcode{as_const_view} cannot be modified.

\pnum
The name \tcode{views::as_const} denotes
a range adaptor object\iref{range.adaptor.object}.
Let \tcode{E} be an expression,
let \tcode{T} be \tcode{decltype((E))}, and
let \tcode{U} be \tcode{remove_cvref_t<T>}.
The expression \tcode{views::as_const(E)} is expression-equivalent to:
\begin{itemize}
\item
If \tcode{views::all_t<T>} models \libconcept{constant_range},
then \tcode{views::all(E)}.
\item
Otherwise,
if \tcode{U} denotes \tcode{empty_view<X>}
for some type \tcode{X}, then \tcode{auto(views::empty<const X>)}.
\begin{addedblock}
\item
Otherwise, if \tcode{U} denotes \tcode{optional<X\&>} for some type \tcode{X} , then \tcode{optional<const X\&>(E)}
\item
Otherwise, if \tcode{U} denotes \tcode{optional<X>} for some type \tcode{X} , then \tcode{optional<const X>(E)}
\end{addedblock}
\item
Otherwise,
if \tcode{U} denotes \tcode{span<X, Extent>}
for some type \tcode{X} and some extent \tcode{Extent},
then \tcode{span<const X, Extent>(E)}.
\item
Otherwise,
if \tcode{U} denotes \tcode{ref_view<X>} for some type \tcode{X} and
\tcode{const X} models \libconcept{constant_range},
then \tcode{ref_view(static_cast<const X\&>(E.base()))}.
\item
Otherwise,
if \tcode{E} is an lvalue,
\tcode{const U} models \libconcept{constant_range}, and
\tcode{U} does not model \libconcept{view},
then \tcode{ref_view(static_cast<const U\&>(E))}.
\item
Otherwise, \tcode{as_const_view(E)}.
\end{itemize}

\rSec2[range.reverse]{Reverse view}

\rSec3[range.reverse.overview]{Overview}

\pnum
\tcode{reverse_view} takes a bidirectional view and produces
another view that iterates the same elements in reverse order.

\pnum
\indexlibrarymember{reverse}{views}%
The name \tcode{views::reverse} denotes a
range adaptor object\iref{range.adaptor.object}.
Given a subexpression \tcode{E}, the expression
\tcode{views::reverse(E)} is expression-equivalent to:
\begin{itemize}
\item
  If the type of \tcode{E} is
  a (possibly cv-qualified) specialization of \tcode{reverse_view},
  then \tcode{E.base()}.
\begin{addedblock}
\item
  Otherwise, If \tcode{T} is specialization of \tcode{optional}, then \placeholdernc{decay-copy}(E).
\end{addedblock}
\item
  Otherwise, if the type of \tcode{E} is \cv{} \tcode{subrange<reverse_iterator<I>, reverse_iterator<I>, K>}
  for some iterator type \tcode{I} and
  value \tcode{K} of type \tcode{subrange_kind},
  \begin{itemize}
  \item
    if \tcode{K} is \tcode{subrange_kind::sized}, then
\tcode{subrange<I, I, K>(E.end().base(), E.begin().base(), E.size())};
  \item
    otherwise, \tcode{subrange<I, I, K>(E.end().base(), E.begin().base())}.
  \end{itemize}
  However, in either case \tcode{E} is evaluated only once.
\item
  Otherwise, \tcode{reverse_view\{E\}}.
\end{itemize}


\end{wording}


\renewcommand{\bibname}{References}
\bibliographystyle{abstract}
\bibliography{wg21,mybiblio}


\end{document}
